\chapter*{Abstract}
\begin{quote}

The port-Hamiltonian systems theory provides a port-based modelling approach, with which complex multiphysical systems can be expressed by interconnection of basic components.  The Exergetic Port-Hamiltonian Systems modeling framework combines the classical port-Hamiltonian systems theory with the GENERIC framework, such that exergetic port-Hamiltonian systems are endowed with structural properties, which imply the first and second law of thermodynamics. In the Exergetic Port-Hamiltonian Systems modeling framework, a system consists of subsystems and the environment, where some subsystems, especially irreversible subsystems, may be unknown. In this thesis, we use structured Neural ODEs and construct an initial value problem. By solving this initial value problem with numerical method, we obtain the predicted state trajectories. We define the loss function by comparing the prediction and the target value. By optimizing the loss, we train neural network models for the unknown subsystems, such that these neural network models can be substituted for the unknown subsystems. Together with the known subsystems, we build a complete model for the system. At the end, we verify that these trained neural network models can be reused for other systems.

\end{quote}