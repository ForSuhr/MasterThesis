\documentclass{article}

% Set language
\usepackage[english]{babel}

% Set page size and margins
\usepackage[a4paper,top=2cm,bottom=2cm,left=3cm,right=3cm,marginparwidth=1.75cm]{geometry}

% Useful package
\usepackage{amsmath}
\usepackage{graphicx}
\usepackage[colorlinks=true, allcolors=blue]{hyperref}
\usepackage{tikz-cd}
\usepackage{multirow}

\title{My Note}
\author{Jiandong Zhao}

\begin{document}
\maketitle

\begin{abstract}
This is a note.
\end{abstract}

\section{Category Theory}

\subsection{Structure of category}

The fundamental structure in category are composition and identity.
\\
\\
Composition: For any pair of morphisms where the first ends where the second starts there exists a composition.
\\
Identity: For any object there exists an identity morphism.

\subsection{Functor}

Functor is the morphisms from one category to another category that preserve the structure of category, i.e. the compositions and identities.
\\
Natural transformation is a morphisms from one functor to another functor that preserve the structure of functor.

\subsection{Acset}
\subsubsection{C-Set}
For a functor $F: C \rightarrow Set$.
\\
Algebra over an operad is a functor from multicategory to Sets, where C is a multicatoegory. Multicategory means multi input single output. The functor F gives concrete meaning to C, in the sense that it picks out some objects(Sets) and morphisms(n-ary Functions).
\subsubsection{C-FinSet}
For a functor $C-FinSet: C \rightarrow FinSet$, where C is a free category and FinSet is a finite Set.
\\
\textbf{C} consists of objects and morphisms:
\begin{itemize}
    \item objects: Box, Port, Junction, BoundaryPort, PortType
    \item morphisms: box, ij, bj, ipt, jpt, bpt
\end{itemize}
\\
\textbf{FinSet} consists of objects and morphisms:
\begin{itemize}
    \item objects: finite sets: set of boxes, set of ports, set of junctions, set of boundaryports, set of port types.
    \item morphisms: functions between finite sets (I will call it sets of mapping), e.g. $~\{ p_1 \rightarrow b_1 , p_1 \rightarrow b_2 , p_2 \rightarrow b_2 \dots ~\}$, $~\{ p_1 \rightarrow j_1 , p_1 \rightarrow j_2 , p_2 \rightarrow j_2 \dots ~\}$, \dots, where $S_f(Port)(p_1)=b_1$ means port 1 belongs to box 1.
\end{itemize}
\\
\textbf{C-Set} is a functor:
\begin{itemize}
    \item In terms of objects: it sends the objects in \textbf{C} to the objects in \textbf{FinSet}, e.g. $S_f(Box) = ~\{ b_1 , b_2 , b_3 \dots ~\}$, $S_f(Port) = ~\{ p_1 , p_2 , p_3 \dots ~\}$, \dots
    \item In terms of morphisms: it sends the morphisms in \textbf{C} to the morphisms in \textbf{FinSet}, e.g. $S_f(box) = ~\{ p_1 \rightarrow b_1 , p_1 \rightarrow b_2 , p_2 \rightarrow b_2 \dots ~\}$, $S_f(ij) = ~\{ p_1 \rightarrow j_1 , p_1 \rightarrow j_2 , p_2 \rightarrow j_2 \dots ~\}$, \dots
\end{itemize}
\\
\subsubsection{Coproduct}
For a functor $\coprod: FinSet \times FinSet \rightarrow FinSet$\\
Coproduct in FinSet or in category is defined by the disjoint union of sets.\\
\subsubsection{Acset}
Attributed C-Set\\

\begin{center}
\begin{tabular}{ |c|c|c|c| } 
\hline
Port & box & ij & ipt \\
\hline
$p_1 & b_1 & \alpha_1 & P_1\\ 
p_2 & b_2 & \alpha_1 & P_1\\ 
p_3 & b_2 & \alpha_2 & P_2$\\ 
\hline
\end{tabular}
\end{center}

\begin{center}
\begin{tabular}{ |c|c|c|c| } 
\hline
Junction & jpt \\
\hline
$\alpha_1 & P_1\\ 
\alpha_2 & P_2$\\ 
\hline
\end{tabular}
\end{center}

\begin{center}
\begin{tabular}{ |c|c|c|c| } 
\hline
BoundaryPort & bj & bpt \\
\hline
$P_4 & \alpha_2 & P_2$\\ 
\hline
\end{tabular}
\end{center}

\subsection{Syntax and Semantics}
Principle of compositionality:\\
The meaning of the whole is determined by the meaning of the parts and how the parts are combined.\\
\\
the whole: the composite system\\
the meaning of the whole: semantics of the composite system\\
the parts: subsystems\\
the meaning of the parts: semantics of the subsystems\\
how the parts are combined: syntax\\
\\
conclusion: The semantics of the composite system is determined by the meaning of the semantics of the subsystems and syntax.\\
\\
Syntax:
\begin{itemize}
    \item objects: all boxes\\
    box is a typed set of ports, e.g. $ b = ~\{p_1:P_1 , p_2:P_2 , \dots~\} $
    \item morphisms: all composition patterns\\
    types correspond to the nature of physical interface
\end{itemize}
\\
There are three types of expressions to represent syntax and semantics: diagrammatic representation(schema, e.g. free category C), mathematical structure(mapping or tables), data structure/data type(in julia code)
\\



\subsection{}


\subsection{}

\end{document}